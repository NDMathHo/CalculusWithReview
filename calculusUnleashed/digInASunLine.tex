\documentclass{ximera}

%\usepackage{todonotes}

\newcommand{\todo}{}

\usepackage{tkz-euclide}
\tikzset{>=stealth} %% cool arrow head
\tikzset{shorten <>/.style={ shorten >=#1, shorten <=#1 } } %% allows shorter vectors

\usepackage{tkz-tab}  %% sign charts
\usetikzlibrary{decorations.pathreplacing} 

\usetikzlibrary{backgrounds} %% for boxes around graphs
\usetikzlibrary{shapes,positioning}  %% Clouds and stars
\usetikzlibrary{matrix} %% for matrix
\usepgfplotslibrary{polar} %% for polar plots
\usetkzobj{all}
\usepackage[makeroom]{cancel} %% for strike outs
%\usepackage{mathtools} %% for pretty underbrace % Breaks Ximera
\usepackage{multicol}

\usepackage{polynom}



\usepackage{array}
\setlength{\extrarowheight}{+.1cm}   
\newdimen\digitwidth
\settowidth\digitwidth{9}
\def\divrule#1#2{
\noalign{\moveright#1\digitwidth
\vbox{\hrule width#2\digitwidth}}}





\newcommand{\RR}{\mathbb R}
\newcommand{\R}{\mathbb R}
\newcommand{\N}{\mathbb N}
\newcommand{\Z}{\mathbb Z}

%\renewcommand{\d}{\,d\!}
\renewcommand{\d}{\mathop{}\!d}
\newcommand{\dd}[2][]{\frac{\d #1}{\d #2}}
\newcommand{\pp}[2][]{\frac{\partial #1}{\partial #2}}
\renewcommand{\l}{\ell}
\newcommand{\ddx}{\frac{d}{\d x}}

\newcommand{\zeroOverZero}{\ensuremath{\boldsymbol{\tfrac{0}{0}}}}
\newcommand{\inftyOverInfty}{\ensuremath{\boldsymbol{\tfrac{\infty}{\infty}}}}
\newcommand{\zeroOverInfty}{\ensuremath{\boldsymbol{\tfrac{0}{\infty}}}}
\newcommand{\zeroTimesInfty}{\ensuremath{\small\boldsymbol{0\cdot \infty}}}
\newcommand{\inftyMinusInfty}{\ensuremath{\small\boldsymbol{\infty - \infty}}}
\newcommand{\oneToInfty}{\ensuremath{\boldsymbol{1^\infty}}}
\newcommand{\zeroToZero}{\ensuremath{\boldsymbol{0^0}}}
\newcommand{\inftyToZero}{\ensuremath{\boldsymbol{\infty^0}}}



\newcommand{\numOverZero}{\ensuremath{\boldsymbol{\tfrac{\#}{0}}}}
\newcommand{\dfn}{\textbf}
%\newcommand{\unit}{\,\mathrm}
\newcommand{\unit}{\mathop{}\!\mathrm}
\newcommand{\eval}[1]{\bigg[ #1 \bigg]}
\newcommand{\seq}[1]{\left( #1 \right)}
\renewcommand{\epsilon}{\varepsilon}
\renewcommand{\iff}{\Leftrightarrow}

\DeclareMathOperator{\arccot}{arccot}
\DeclareMathOperator{\arcsec}{arcsec}
\DeclareMathOperator{\arccsc}{arccsc}
\DeclareMathOperator{\si}{Si}
\DeclareMathOperator{\proj}{proj}
\DeclareMathOperator{\scal}{scal}


\newcommand{\tightoverset}[2]{% for arrow vec
  \mathop{#2}\limits^{\vbox to -.5ex{\kern-0.75ex\hbox{$#1$}\vss}}}
\newcommand{\arrowvec}[1]{\tightoverset{\scriptstyle\rightharpoonup}{#1}}
\renewcommand{\vec}{\mathbf}
\newcommand{\veci}{\vec{i}}
\newcommand{\vecj}{\vec{j}}
\newcommand{\veck}{\vec{k}}
\newcommand{\vecl}{\boldsymbol{\l}}

\newcommand{\dotp}{\bullet}
\newcommand{\cross}{\boldsymbol\times}
\newcommand{\grad}{\boldsymbol\nabla}
\newcommand{\divergence}{\grad\dotp}
\newcommand{\curl}{\grad\cross}
%\DeclareMathOperator{\divergence}{divergence}
%\DeclareMathOperator{\curl}[1]{\grad\cross #1}


\colorlet{textColor}{black} 
\colorlet{background}{white}
\colorlet{penColor}{blue!50!black} % Color of a curve in a plot
\colorlet{penColor2}{red!50!black}% Color of a curve in a plot
\colorlet{penColor3}{red!50!blue} % Color of a curve in a plot
\colorlet{penColor4}{green!50!black} % Color of a curve in a plot
\colorlet{penColor5}{orange!80!black} % Color of a curve in a plot
\colorlet{fill1}{penColor!20} % Color of fill in a plot
\colorlet{fill2}{penColor2!20} % Color of fill in a plot
\colorlet{fillp}{fill1} % Color of positive area
\colorlet{filln}{penColor2!20} % Color of negative area
\colorlet{fill3}{penColor3!20} % Fill
\colorlet{fill4}{penColor4!20} % Fill
\colorlet{fill5}{penColor5!20} % Fill
\colorlet{gridColor}{gray!50} % Color of grid in a plot

\newcommand{\surfaceColor}{violet}
\newcommand{\surfaceColorTwo}{redyellow}
\newcommand{\sliceColor}{greenyellow}




\pgfmathdeclarefunction{gauss}{2}{% gives gaussian
  \pgfmathparse{1/(#2*sqrt(2*pi))*exp(-((x-#1)^2)/(2*#2^2))}%
}


%%%%%%%%%%%%%
%% Vectors
%%%%%%%%%%%%%

%% Simple horiz vectors
\renewcommand{\vector}[1]{\left\langle #1\right\rangle}


%% %% Complex Horiz Vectors with angle brackets
%% \makeatletter
%% \renewcommand{\vector}[2][ , ]{\left\langle%
%%   \def\nextitem{\def\nextitem{#1}}%
%%   \@for \el:=#2\do{\nextitem\el}\right\rangle%
%% }
%% \makeatother

%% %% Vertical Vectors
%% \def\vector#1{\begin{bmatrix}\vecListA#1,,\end{bmatrix}}
%% \def\vecListA#1,{\if,#1,\else #1\cr \expandafter \vecListA \fi}

%%%%%%%%%%%%%
%% End of vectors
%%%%%%%%%%%%%

%\newcommand{\fullwidth}{}
%\newcommand{\normalwidth}{}



%% makes a snazzy t-chart for evaluating functions
%\newenvironment{tchart}{\rowcolors{2}{}{background!90!textColor}\array}{\endarray}

%%This is to help with formatting on future title pages.
\newenvironment{sectionOutcomes}{}{} 



%% Flowchart stuff
%\tikzstyle{startstop} = [rectangle, rounded corners, minimum width=3cm, minimum height=1cm,text centered, draw=black]
%\tikzstyle{question} = [rectangle, minimum width=3cm, minimum height=1cm, text centered, draw=black]
%\tikzstyle{decision} = [trapezium, trapezium left angle=70, trapezium right angle=110, minimum width=3cm, minimum height=1cm, text centered, draw=black]
%\tikzstyle{question} = [rectangle, rounded corners, minimum width=3cm, minimum height=1cm,text centered, draw=black]
%\tikzstyle{process} = [rectangle, minimum width=3cm, minimum height=1cm, text centered, draw=black]
%\tikzstyle{decision} = [trapezium, trapezium left angle=70, trapezium right angle=110, minimum width=3cm, minimum height=1cm, text centered, draw=black]



\title[Dig-In:]{A Sun-line}

\begin{document}
\begin{abstract}
  Here we ``play'' with mathematics.
\end{abstract}
\maketitle


\section{A mission gone ary}

You are a top-secret agent, on a mission that puts you over the
Pacific on a plane flying nearly over the equator. Though you are the
world's most respected secret-angent, you had no idea what was about
to happen next.

There you were, being your (frankly awesome) spy-self. When ``some
nonsense'' turned into the ``some other nonsense'' and the next thing
you knew, you were tossed out of the plane plunging downward into the
dark Pacfic. Never to loose hope, you spread your arms and legs to
increase your surface area, and hence atmospheric drag.

%% https://www.grc.nasa.gov/www/k-12/airplane/termv.html

What would you need to reduce your frontal area to (drag coefficient?) to ??? 


Through a clever use of nearly all of your garments, you \textit{are}
able to increase your frontal area (and drag coefficient?) to exactly the necessary value to reduce your impact speed to....

kE?

As you splash down into the chilly waters of the Pacific, you see a
small desert island nearly 500 m ahead of you...

%% optimization with currents? swimming if your island is on a path
%% y=x, you could compute the inverse function to a account for currents...

With your last bit of strength, you manage to swim to the island, and
finally know the ressurance of solid ground. As you lay in the warm
sand, you realize that you are a pretty awesome secret agent.


\section{The next morning}

The next morning you awake and immediatley start to concot a plan to
get your off this island and back to mission.


Hidden beneith your skin, is a secret, single-use, transponder
beacon. If ever you are in trouble in the world, you can, through a
very careful application of pressure, trigger this transponder. It
will be active for a mere five seconds. However, if you set off the beacon at exactly 11:51 pm GMT


MAYBE YOU HAVE THE ABILITY TO GET 1 time, and MAKE One call. 




TRAPPED ON GALAPAGOSE ISLAND - NEED A CLOCK - USE LINEAR APPROX


Imagine that the world was flat and that everyday, the Sun (infinitely
far away) acutally rises up at $6\unit{am}$ in the East, and sets at
$6\unit{pm}$ in the west.
  \begin{image}
    \begin{tikzpicture}
      \begin{axis}[
          xmin=-1.1,xmax=1.1,ymin=-1.1,ymax=1.1,
          axis lines=center,
          width=4in,
          ticks=none,
          clip=false,
          unit vector ratio*=1 1 1,
          %xlabel=$x$, ylabel=$y$,
          every axis y label/.style={at=(current axis.above origin),anchor=south},
          every axis x label/.style={at=(current axis.right of origin),anchor=west},
        ]        
        \addplot [penColor2!50!white, very thick, smooth, domain=(-90:90)] ({cos(x)},{sin(x)}); %% unit circle
%        \addplot [black!50!white, dashed, smooth, domain=(90:270)] ({cos(x)},{sin(x)}); %% unit circle
        \addplot[color=penColor2!50!white,fill=penColor2!50!white,only marks,mark=*] coordinates{(0,1)};  %% closed hole
        \addplot[color=penColor2!50!white,fill=penColor2!50!white,only marks,mark=*] coordinates{(0,-1)};  %% closed hole         
        \addplot [ultra thick] plot coordinates {(0,0) (.766,.643)}; %% 40 degrees
        
        \addplot [ultra thick] plot coordinates {(.766,0) (.766,.643)}; %% 40 degrees
        \addplot [ultra thick] plot coordinates {(0,0) (.766,0)}; %% 40 degrees
        
        %\addplot [ultra thick,penColor3] plot coordinates {(1,0) (1,.839)}; %% 40 degrees          
        
        \addplot [textColor,smooth, domain=(0:40)] ({.15*cos(x)},{.15*sin(x)});
        %\addplot [very thick,penColor] plot coordinates {(0,0) (.766,.643)}; %% sector
        %\addplot [very thick,penColor] plot coordinates {(0,0) (1,0)}; %% sector
        %\addplot [very thick, penColor, smooth, domain=(0:40)] ({cos(x)},{sin(x)}); %% sector
        \node at (axis cs:.12,.07) [anchor=west] {$\theta$};
        \node at (axis cs:.84,.322) {$x$};
        \node at (axis cs:.38,.32) [anchor=south] {$1$};
      \end{axis}
    \end{tikzpicture}
  \end{image}
In the diagram above, we see a rod of height $h$ casting a shadow. We
can see that the slope of the line connecting the point representing
the Sun to the tip of the rod is
\[
\frac{R\sin\left(\frac{n\pi}{12}\right)-h}{R\cos\left(\frac{n\pi}{12}\right)}
\]
If we take the limit as $R$ goes to infinity, then we find,
\begin{align*}
  \lim_{R\to \infty} \frac{R\sin\left(\frac{n\pi}{12}\right)-h}{R\cos\left(\frac{n\pi}{12}\right)} &=\lim_{R\to \infty}\frac{R\sin\left(\frac{n\pi}{12}\right)-h}{R\cos\left(\frac{n\pi}{12}\right)}\cdot\frac{1/R}{1/R}\\
  &= \lim_{R\to \infty}\frac{\sin\left(\frac{n\pi}{12}\right)-h/R}{\cos\left(\frac{n\pi}{12}\right)}\\
  &= \frac{\sin\left(\frac{n\pi}{12}\right)}{\cos\left(\frac{n\pi}{12}\right)}\\
  &= \tan\left(\frac{n\pi}{12}\right)
\end{align*}

\section{What does the graph look like?}

We'll use all of our curve sketching techniques to try to understand
this function on the interval $[-2\pi,2\pi]$, as gesture of
friendship, we'll tell you that $\si(x)$ is continuous on
$[-2\pi,2\pi]$.  The first thing we should do is compute the first
derivative of $\si(x)$.

\begin{example}
  Compute:
  \[
  \ddx \si(x)
  \]
  \begin{explanation}
    By the First Fundamental Theorem of Calculus,
    \begin{align*}
    \ddx \si(x) &= \ddx \int_0^x \frac{\sin(t)}{t}\d t\\
    &=\frac{\sin(x)}{x}.
    \end{align*}
  \end{explanation}
\end{example}

Now we'll compute the second derivative of $\si(x)$.

\begin{example}
  Compute:
  \[
  \dd[^2]{x^2}\si(x)
  \]
  \begin{explanation}
    By our previous work and the quotient rule we see
    \begin{align*}
      \dd[^2]{x^2} \si(x) &= \ddx \frac{\sin(x)}{x}\\
      &=\frac{x\cos(x) -\sin(x)}{x^2}.
    \end{align*}
  \end{explanation}
\end{example}

Now we should find the $y$-intercept.

\begin{example}
  Compute $\si(0)$.
  \begin{explanation}
    Here $\si(0)=0$, as $\si$ is an accumulation
    function, and at $x=0$, no area has been accumulated.
  \end{explanation}
\end{example}

%% Vertical asymptotes

%% end behavior

Now we'll look for critical points, where the derivative is zero or
undefined.

\begin{example}
  Find the critical points of $\si(x)$.
  \begin{explanation}
    The critical points are where $\si'(x)= 0$ or it does not exist.
    Since
    \[
    \si'(x) = \frac{\sin(x)}{x}.
    \]
    We see that this derivative does not exist at $x=0$, and for $x\ne
    0$, $\si'(x)=0$ precisely when $\sin(x)$ is zero. Since $\sin(x)$
    is zero at $x = -2\pi, -\pi, \pi, 2\pi$, we see that the critical
    points are where
    \[
    x= -2\pi, -\pi,0, \pi, 2\pi.
    \]
  \end{explanation}
\end{example}

We'll identify which of these are maximums and minimums.

\begin{example}
  Find the local extrema of $\si(x)$ on the interval $[-2\pi,2\pi]$.
  \begin{explanation}
    The critical points are at
    \[
    x= -2\pi, -\pi,0, \pi, 2\pi.
    \]
    We will use the first derivative test to identifiy which of these
    are local extrema.
    
  \end{explanation}
\end{example}

Inflection points are harder. Let's try our hand.

\begin{example}
  Find the inflection points of $\si(x)$.
  \begin{explanation}
    We start by looking at the second derivative of $\si(x)$,
    \[
    \si''(x) = \frac{x\cos(x) -\sin(x)}{x^2}.
    \]
    The first candidate for an infection point is $x=0$, since the
    second derivative does not exist. To find other inflection points
    on $[-2\pi,2\pi]$, we need to find when
    \[
    x\cos(x) -\sin(x) =0.
    \]
    This is zero when $x=0$, anywhere else?
  \end{explanation}
\end{example}



\end{document}
