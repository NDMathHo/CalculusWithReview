\documentclass{ximera}

%\usepackage{todonotes}

\newcommand{\todo}{}

\usepackage{tkz-euclide}
\tikzset{>=stealth} %% cool arrow head
\tikzset{shorten <>/.style={ shorten >=#1, shorten <=#1 } } %% allows shorter vectors

\usepackage{tkz-tab}  %% sign charts
\usetikzlibrary{decorations.pathreplacing} 

\usetikzlibrary{backgrounds} %% for boxes around graphs
\usetikzlibrary{shapes,positioning}  %% Clouds and stars
\usetikzlibrary{matrix} %% for matrix
\usepgfplotslibrary{polar} %% for polar plots
\usetkzobj{all}
\usepackage[makeroom]{cancel} %% for strike outs
%\usepackage{mathtools} %% for pretty underbrace % Breaks Ximera
\usepackage{multicol}

\usepackage{polynom}



\usepackage{array}
\setlength{\extrarowheight}{+.1cm}   
\newdimen\digitwidth
\settowidth\digitwidth{9}
\def\divrule#1#2{
\noalign{\moveright#1\digitwidth
\vbox{\hrule width#2\digitwidth}}}





\newcommand{\RR}{\mathbb R}
\newcommand{\R}{\mathbb R}
\newcommand{\N}{\mathbb N}
\newcommand{\Z}{\mathbb Z}

%\renewcommand{\d}{\,d\!}
\renewcommand{\d}{\mathop{}\!d}
\newcommand{\dd}[2][]{\frac{\d #1}{\d #2}}
\newcommand{\pp}[2][]{\frac{\partial #1}{\partial #2}}
\renewcommand{\l}{\ell}
\newcommand{\ddx}{\frac{d}{\d x}}

\newcommand{\zeroOverZero}{\ensuremath{\boldsymbol{\tfrac{0}{0}}}}
\newcommand{\inftyOverInfty}{\ensuremath{\boldsymbol{\tfrac{\infty}{\infty}}}}
\newcommand{\zeroOverInfty}{\ensuremath{\boldsymbol{\tfrac{0}{\infty}}}}
\newcommand{\zeroTimesInfty}{\ensuremath{\small\boldsymbol{0\cdot \infty}}}
\newcommand{\inftyMinusInfty}{\ensuremath{\small\boldsymbol{\infty - \infty}}}
\newcommand{\oneToInfty}{\ensuremath{\boldsymbol{1^\infty}}}
\newcommand{\zeroToZero}{\ensuremath{\boldsymbol{0^0}}}
\newcommand{\inftyToZero}{\ensuremath{\boldsymbol{\infty^0}}}



\newcommand{\numOverZero}{\ensuremath{\boldsymbol{\tfrac{\#}{0}}}}
\newcommand{\dfn}{\textbf}
%\newcommand{\unit}{\,\mathrm}
\newcommand{\unit}{\mathop{}\!\mathrm}
\newcommand{\eval}[1]{\bigg[ #1 \bigg]}
\newcommand{\seq}[1]{\left( #1 \right)}
\renewcommand{\epsilon}{\varepsilon}
\renewcommand{\iff}{\Leftrightarrow}

\DeclareMathOperator{\arccot}{arccot}
\DeclareMathOperator{\arcsec}{arcsec}
\DeclareMathOperator{\arccsc}{arccsc}
\DeclareMathOperator{\si}{Si}
\DeclareMathOperator{\proj}{proj}
\DeclareMathOperator{\scal}{scal}


\newcommand{\tightoverset}[2]{% for arrow vec
  \mathop{#2}\limits^{\vbox to -.5ex{\kern-0.75ex\hbox{$#1$}\vss}}}
\newcommand{\arrowvec}[1]{\tightoverset{\scriptstyle\rightharpoonup}{#1}}
\renewcommand{\vec}{\mathbf}
\newcommand{\veci}{\vec{i}}
\newcommand{\vecj}{\vec{j}}
\newcommand{\veck}{\vec{k}}
\newcommand{\vecl}{\boldsymbol{\l}}

\newcommand{\dotp}{\bullet}
\newcommand{\cross}{\boldsymbol\times}
\newcommand{\grad}{\boldsymbol\nabla}
\newcommand{\divergence}{\grad\dotp}
\newcommand{\curl}{\grad\cross}
%\DeclareMathOperator{\divergence}{divergence}
%\DeclareMathOperator{\curl}[1]{\grad\cross #1}


\colorlet{textColor}{black} 
\colorlet{background}{white}
\colorlet{penColor}{blue!50!black} % Color of a curve in a plot
\colorlet{penColor2}{red!50!black}% Color of a curve in a plot
\colorlet{penColor3}{red!50!blue} % Color of a curve in a plot
\colorlet{penColor4}{green!50!black} % Color of a curve in a plot
\colorlet{penColor5}{orange!80!black} % Color of a curve in a plot
\colorlet{fill1}{penColor!20} % Color of fill in a plot
\colorlet{fill2}{penColor2!20} % Color of fill in a plot
\colorlet{fillp}{fill1} % Color of positive area
\colorlet{filln}{penColor2!20} % Color of negative area
\colorlet{fill3}{penColor3!20} % Fill
\colorlet{fill4}{penColor4!20} % Fill
\colorlet{fill5}{penColor5!20} % Fill
\colorlet{gridColor}{gray!50} % Color of grid in a plot

\newcommand{\surfaceColor}{violet}
\newcommand{\surfaceColorTwo}{redyellow}
\newcommand{\sliceColor}{greenyellow}




\pgfmathdeclarefunction{gauss}{2}{% gives gaussian
  \pgfmathparse{1/(#2*sqrt(2*pi))*exp(-((x-#1)^2)/(2*#2^2))}%
}


%%%%%%%%%%%%%
%% Vectors
%%%%%%%%%%%%%

%% Simple horiz vectors
\renewcommand{\vector}[1]{\left\langle #1\right\rangle}


%% %% Complex Horiz Vectors with angle brackets
%% \makeatletter
%% \renewcommand{\vector}[2][ , ]{\left\langle%
%%   \def\nextitem{\def\nextitem{#1}}%
%%   \@for \el:=#2\do{\nextitem\el}\right\rangle%
%% }
%% \makeatother

%% %% Vertical Vectors
%% \def\vector#1{\begin{bmatrix}\vecListA#1,,\end{bmatrix}}
%% \def\vecListA#1,{\if,#1,\else #1\cr \expandafter \vecListA \fi}

%%%%%%%%%%%%%
%% End of vectors
%%%%%%%%%%%%%

%\newcommand{\fullwidth}{}
%\newcommand{\normalwidth}{}



%% makes a snazzy t-chart for evaluating functions
%\newenvironment{tchart}{\rowcolors{2}{}{background!90!textColor}\array}{\endarray}

%%This is to help with formatting on future title pages.
\newenvironment{sectionOutcomes}{}{} 



%% Flowchart stuff
%\tikzstyle{startstop} = [rectangle, rounded corners, minimum width=3cm, minimum height=1cm,text centered, draw=black]
%\tikzstyle{question} = [rectangle, minimum width=3cm, minimum height=1cm, text centered, draw=black]
%\tikzstyle{decision} = [trapezium, trapezium left angle=70, trapezium right angle=110, minimum width=3cm, minimum height=1cm, text centered, draw=black]
%\tikzstyle{question} = [rectangle, rounded corners, minimum width=3cm, minimum height=1cm,text centered, draw=black]
%\tikzstyle{process} = [rectangle, minimum width=3cm, minimum height=1cm, text centered, draw=black]
%\tikzstyle{decision} = [trapezium, trapezium left angle=70, trapezium right angle=110, minimum width=3cm, minimum height=1cm, text centered, draw=black]


\title[Dig-In:]{L'H\^{o}pital's rule}

\outcome{Recall how to find limits for forms that are not indeterminate.}
\outcome{Define an indeterminate form.}
\outcome{Convert indeterminate forms to the form zero over zero or infinity over infinity.}
\outcome{Define L'Hopital's Rule and identify when it can be used.}
\outcome{Use L'Hopital's Rule to find limits.}

\begin{document}
\begin{abstract}
  We use derivatives to give us a ``short-cut'' for computing limits.
\end{abstract}
\maketitle

Derivatives allow us to take problems that were once difficult to
solve and convert them to problems that are easier to solve. Let us
consider L'H\^{o}pital's rule:

\begin{theorem}[L'H\^{o}pital's Rule]\index{L'H\^opital's Rule} 
Let $f(x)$ and $g(x)$ be functions that are differentiable near $a$.  If
\[
\lim_{x \to a} f(x) = \lim_{x \to a}g(x) = 0 \qquad \text{or} \pm \infty,
\]
and $\lim_{x \to a} \frac{f'(x)}{g'(x)}$ exists, and $g'(x) \neq 0$
for all $x$ near $a$, then 
\[
\lim_{x \to a} \frac{f(x)}{g(x)} = \lim_{x \to a} \frac{f'(x)}{g'(x)}.
\]
\end{theorem}

This theorem is somewhat difficult to prove, in part because it
incorporates so many different possibilities, so we will not prove it
here. 
\begin{remark}
  L'H\^{o}pital's rule applies even when $\lim_{x\to a}f(x) = \pm \infty$
  and $\lim_{x\to a}g(x) = \mp \infty$.
\end{remark}


L'H\^{o}pital's rule allows us to investigate limits of
\textit{indeterminate form}.

\begin{definition}[List of Indeterminate Forms]\index{indeterminate form}\hfil
\begin{description}
\item[\zeroOverZero] This refers to a limit of the form $\lim_{x\to a}
  \frac{f(x)}{g(x)}$ where $f(x)\to 0$ and $g(x)\to 0$ as $x\to a$.
\item[\inftyOverInfty] This refers to a limit of the form $\lim_{x\to a}
  \frac{f(x)}{g(x)}$ where $f(x)\to \infty$ and $g(x)\to \infty$ as $x\to a$.
\item[\zeroTimesInfty] This refers to a limit of the form $\lim_{x\to a}
  \left(f(x)\cdot g(x)\right)$ where $f(x)\to 0$ and $g(x)\to \infty$ as $x\to a$.
\item[\inftyMinusInfty] This refers to a limit of the form $\lim_{x\to a}\left(
  f(x)-g(x)\right)$ where $f(x)\to \infty$ and $g(x)\to \infty$ as $x\to a$.
\item[\oneToInfty] This refers to a limit of the form $\lim_{x\to a}
  f(x)^{g(x)}$ where $f(x)\to 1$ and $g(x)\to \infty$ as $x\to a$.
\item[\zeroToZero] This refers to a limit of the form $\lim_{x\to a}
  f(x)^{g(x)}$ where $f(x)\to 0$ and $g(x)\to 0$ as $x\to a$.
\item[\inftyToZero] This refers to a limit of the form $\lim_{x\to a}
  f(x)^{g(x)}$ where $f(x)\to \infty$ and $g(x)\to 0$ as $x\to a$.
\end{description}
In each of these cases, the value of the limit is \textbf{not} immediately
obvious. Hence, a careful analysis is required!
\end{definition}

\section{Basic indeterminant forms}


Our first example is the computation of a limit that was somewhat
difficult before.

\begin{example}
Compute
\[
\lim_{x\to 0} \frac{\sin(x)}{x}.
\]
\begin{explanation}
Set $f(x) = \sin(x)$ and $g(x) = x$.  Since both $f(x)$ and $g(x)$ are
differentiable functions at $0$, and 
\[
\lim_{x \to 0} f(x) = \lim_{x \to 0}g(x) = 0,
\]
this situation is ripe for L'H\^{o}pital's Rule. Now
\[
f'(x) = \answer[given]{\cos(x)}
\]
and
\[
g'(x) = \answer[given]{1}.
\] 
L'H\^{o}pital's rule tells us that 
\[
\lim_{x \to 0} \frac{\sin(x)}{x} = \lim_{x \to 0} \frac{\cos(x)}{1} = 1.
\]
\end{explanation}
\end{example}

\begin{remark}
  Note, the astute mathematician will notice that in our example
  above, we are somewhat cheating. To apply L'H\^ opital's rule, we
  need to know the derivative of sine; however, to know the derivative
  of sine we must be able to compute the limit:
  \[
  \lim_{x\to 0}\frac{\sin(x)}{x}
  \]
  Hence using L'H\^{o}pital's rule to compute this limit is a circular
  argument! We encourage the gentle reader to view L'H\^{o}pital's rule
  a ``reminder'' as to what is true, not as the formal derivation of
  the result.
\end{remark}


Our next set of examples will run through the remaining indeterminate
forms one is likely to encounter.

\begin{example}
  Compute 
\[
\lim_{x\to \pi/2^+} \frac{\sec(x)}{\tan(x)}.
\]
\begin{explanation}
Set $f(x) = \sec(x)$ and $g(x) = \tan(x)$. Both $f(x)$ and $g(x)$
are differentiable near $\pi/2$. Additionally,
\[
\lim_{x \to \pi/2^+} f(x) = \lim_{x \to \pi/2^+}g(x) = -\infty.
\]
This situation is ripe for L'H\^opital's Rule. Now 
\[
f'(x) = \answer[given]{\sec(x)\tan(x)}
\]
and
\[
g'(x) = \answer[given]{\sec^2(x)}.
\]
L'H\^{o}pital's rule tells us that 
\begin{align*}
\lim_{x\to \pi/2^+} \frac{\sec(x)}{\tan(x)} &= \lim_{x\to \pi/2^+}
\frac{\sec(x)\tan(x)}{\sec^2(x)} \\
&= \lim_{x\to \pi/2^+} \sin(x)\\
&=\answer[given]{1}.
\end{align*}
\end{explanation}
\end{example}



\begin{example}\label{example:xlnx infty} 
Compute 
\[
\lim_{x\to 0^+} x\ln x.
\]
\begin{explanation}
This doesn't appear to be suitable for L'H\^{o}pital's Rule. As $x$
approaches zero, $\ln x$ goes to $-\infty$, so the product looks like
\[
(\text{something very small})\cdot (\text{something very large and
  negative}).
\] 
This product could be anything. A careful analysis is required.
Write
\[
x\ln x = \frac{\ln x}{x^{-1}}.
\]
Set $f(x) = \ln(x)$ and $g(x) = x^{-1}$.  Since both functions are differentiable near zero and 
\[
\lim_{x\to 0+} \ln(x) = -\infty\qquad\text{and}\qquad \lim_{x\to 0+} x^{-1} = \infty,
\]
we may apply L'H\^{o}pital's rule. Write with me
\[
f'(x) = \answer[given]{x^{-1}}
\]
and
\[
g'(x) = \answer[given]{-x^{-2}},
\]
so
\begin{align*}
  \lim_{x\to 0^+} x\ln x &= \lim_{x\to 0^+} \frac{\ln x}{x^{-1}} \\
  &= \lim_{x\to 0^+} \frac{x^{-1}}{-x^{-2}}\\
  &=\lim_{x\to 0^+} -x \\
  &= 0.
\end{align*}
One way to interpret this is that since $\lim_{x\to 0^+}x\ln x = 0$,
the function $x$ approaches zero much faster than $\ln x$ approaches
$-\infty$.
\end{explanation}
\end{example}


\section{Indeterminate forms involving subtraction}

There are two basic cases here, we'll do an example of each.

\begin{example}
Compute
\[
\lim_{x\to 0} \left(\cot(x) - \csc(x)\right).
\]
\begin{explanation}
Here we simply need to write each term as a fraction,
\begin{align*}
\lim_{x\to 0} \left(\cot(x) - \csc(x)\right) &= \lim_{x\to 0} \left(\frac{\cos(x)}{\sin(x)} - \frac{1}{\sin(x)}\right)\\
&= \lim_{x\to 0} \frac{\cos(x)-1}{\sin(x)} 
\end{align*}
Setting $f(x) = \cos(x)-1$ and $g(x)=\sin(x)$, both functions are differentiable near zero and 
\[
\lim_{x\to 0}(\cos(x)-1)=\lim_{x\to 0}\sin(x) = 0.
\]
We may now apply L'H\^{o}pital's rule. Write with me
\[
f'(x) = \answer[given]{-\sin(x)}
\]
and
\[
g'(x) = \answer[given]{\cos(x)},
\]
so
\begin{align*}
  \lim_{x\to 0} \left(\cot(x) - \csc(x)\right) &= \lim_{x\to 0} \frac{\cos(x)-1}{\sin(x)}\\
  &= \lim_{x\to 0} \frac{-\sin(x)}{\cos(x)} \\
  &=0.
\end{align*}
\end{explanation}
\end{example}


Sometimes one must be slightly more clever. 

\begin{example}
Compute
\[
\lim_{x\to\infty}\left(\sqrt{x^2+x}-x\right).
\]
\begin{explanation}
Again, this doesn't appear to be suitable for L'H\^{o}pital's Rule. A
bit of algebraic manipulation will help. Write with me
\begin{align*}
\lim_{x\to\infty}\left(\sqrt{x^2+x}-x\right) &= \lim_{x\to\infty}\left(x\left(\sqrt{1+1/x}-1\right)\right)\\
&=\lim_{x\to\infty}\frac{\sqrt{1+1/x}-1}{x^{-1}}
\end{align*}
Now set $f(x) = \sqrt{1+1/x}-1$, $g(x) = x^{-1}$. Since both
  functions are differentiable for large values of $x$ and 
\[
\lim_{x\to\infty} (\sqrt{1+1/x}-1) = \lim_{x\to\infty}x^{-1} = 0, 
\]
we may apply L'H\^{o}pital's rule. Write with me
\[
f'(x) = \answer[given]{(1/2)(1+1/x)^{-1/2}\cdot(-x^{-2})}
\]
and
\[
g'(x) = \answer[given]{-x^{-2}}
\]
so
\begin{align*}
\lim_{x\to\infty}\left(\sqrt{x^2+x}-x\right) &= \lim_{x\to\infty}\frac{\sqrt{1+1/x}-1}{x^{-1}} \\
&= \lim_{x\to\infty}\frac{(1/2)(1+1/x)^{-1/2}\cdot(-x^{-2})}{-x^{-2}} \\
&= \lim_{x\to\infty} \frac{1}{2\sqrt{1+1/x}}\\
&= \frac{1}{2}.
\end{align*}
\end{explanation}
\end{example}


\section{Exponential Indeterminate Forms}

There is a standard trick for dealing with the indeterminate forms
\[
\text{\oneToInfty},\quad \text{\zeroToZero},\quad \text{\inftyToZero}.
\]
Given $u(x)$ and $v(x)$ such that
\[
\lim_{x\to a}u(x)^{v(x)}
\]
falls into one of the categories described above, rewrite as
\[
\lim_{x\to a}e^{v(x)\ln(u(x))}
\]
and then examine the limit of the exponent
\[
\lim_{x\to a} v(x)\ln(u(x)) = \lim_{x\to a} \frac{\ln(u(x))}{v(x)^{-1}}
\]
using L'H\^{o}pital's rule.  Since these forms are all very similar, we
will only give a single example.


\begin{example}
Compute
\[
\lim_{x\to \infty}\left(1 + \frac{1}{x}\right)^x.
\]
\begin{explanation}
Write
\[
\lim_{x\to \infty}\left(1 + \frac{1}{x}\right)^x = \lim_{x\to \infty}e^{x\ln\left(1 + \frac{1}{x}\right)}.
\]
So now look at the limit of the exponent
\[
\lim_{x\to\infty} x\ln\left(1 + \frac{1}{x}\right) = \lim_{x\to\infty} \frac{\ln\left(1 + \frac{1}{x}\right)}{x^{-1}}.
\]
Setting $f(x) = \ln\left(1 + \frac{1}{x}\right)$ and $g(x) = x^{-1}$,
both functions are differentiable for large values of $x$ and
\[
\lim_{x\to \infty}\ln\left(1 + \frac{1}{x}\right)=\lim_{x\to \infty}x^{-1} = 0.
\]
We may now apply L'H\^{o}pital's rule. Write
\[
f'(x) = \answer[given]{\frac{-x^{-2}}{1 + \frac{1}{x}}}
\]
and
\[
g'(x) = \answer[given]{-x^{-2}},
\]
so
\begin{align*}
\lim_{x\to\infty} \frac{\ln\left(1 + \frac{1}{x}\right)}{x^{-1}} &= \lim_{x\to\infty} \frac{\frac{-x^{-2}}{1 + \frac{1}{x}}}{-x^{-2}} \\
&=\lim_{x\to\infty} \frac{1}{1 + \frac{1}{x}}\\
&=1.
\end{align*}
Hence, 
\[
\lim_{x\to \infty}\left(1 + \frac{1}{x}\right)^x = \lim_{x\to \infty}e^{x\ln\left(1 + \frac{1}{x}\right)} =e^{1} = e.
\]
\end{explanation}
\end{example}





\end{document}

