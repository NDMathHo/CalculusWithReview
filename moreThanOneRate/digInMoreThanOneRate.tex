\documentclass{ximera}

%\usepackage{todonotes}

\newcommand{\todo}{}

\usepackage{tkz-euclide}
\tikzset{>=stealth} %% cool arrow head
\tikzset{shorten <>/.style={ shorten >=#1, shorten <=#1 } } %% allows shorter vectors

\usepackage{tkz-tab}  %% sign charts
\usetikzlibrary{decorations.pathreplacing} 

\usetikzlibrary{backgrounds} %% for boxes around graphs
\usetikzlibrary{shapes,positioning}  %% Clouds and stars
\usetikzlibrary{matrix} %% for matrix
\usepgfplotslibrary{polar} %% for polar plots
\usetkzobj{all}
\usepackage[makeroom]{cancel} %% for strike outs
%\usepackage{mathtools} %% for pretty underbrace % Breaks Ximera
\usepackage{multicol}

\usepackage{polynom}



\usepackage{array}
\setlength{\extrarowheight}{+.1cm}   
\newdimen\digitwidth
\settowidth\digitwidth{9}
\def\divrule#1#2{
\noalign{\moveright#1\digitwidth
\vbox{\hrule width#2\digitwidth}}}





\newcommand{\RR}{\mathbb R}
\newcommand{\R}{\mathbb R}
\newcommand{\N}{\mathbb N}
\newcommand{\Z}{\mathbb Z}

%\renewcommand{\d}{\,d\!}
\renewcommand{\d}{\mathop{}\!d}
\newcommand{\dd}[2][]{\frac{\d #1}{\d #2}}
\newcommand{\pp}[2][]{\frac{\partial #1}{\partial #2}}
\renewcommand{\l}{\ell}
\newcommand{\ddx}{\frac{d}{\d x}}

\newcommand{\zeroOverZero}{\ensuremath{\boldsymbol{\tfrac{0}{0}}}}
\newcommand{\inftyOverInfty}{\ensuremath{\boldsymbol{\tfrac{\infty}{\infty}}}}
\newcommand{\zeroOverInfty}{\ensuremath{\boldsymbol{\tfrac{0}{\infty}}}}
\newcommand{\zeroTimesInfty}{\ensuremath{\small\boldsymbol{0\cdot \infty}}}
\newcommand{\inftyMinusInfty}{\ensuremath{\small\boldsymbol{\infty - \infty}}}
\newcommand{\oneToInfty}{\ensuremath{\boldsymbol{1^\infty}}}
\newcommand{\zeroToZero}{\ensuremath{\boldsymbol{0^0}}}
\newcommand{\inftyToZero}{\ensuremath{\boldsymbol{\infty^0}}}



\newcommand{\numOverZero}{\ensuremath{\boldsymbol{\tfrac{\#}{0}}}}
\newcommand{\dfn}{\textbf}
%\newcommand{\unit}{\,\mathrm}
\newcommand{\unit}{\mathop{}\!\mathrm}
\newcommand{\eval}[1]{\bigg[ #1 \bigg]}
\newcommand{\seq}[1]{\left( #1 \right)}
\renewcommand{\epsilon}{\varepsilon}
\renewcommand{\iff}{\Leftrightarrow}

\DeclareMathOperator{\arccot}{arccot}
\DeclareMathOperator{\arcsec}{arcsec}
\DeclareMathOperator{\arccsc}{arccsc}
\DeclareMathOperator{\si}{Si}
\DeclareMathOperator{\proj}{proj}
\DeclareMathOperator{\scal}{scal}


\newcommand{\tightoverset}[2]{% for arrow vec
  \mathop{#2}\limits^{\vbox to -.5ex{\kern-0.75ex\hbox{$#1$}\vss}}}
\newcommand{\arrowvec}[1]{\tightoverset{\scriptstyle\rightharpoonup}{#1}}
\renewcommand{\vec}{\mathbf}
\newcommand{\veci}{\vec{i}}
\newcommand{\vecj}{\vec{j}}
\newcommand{\veck}{\vec{k}}
\newcommand{\vecl}{\boldsymbol{\l}}

\newcommand{\dotp}{\bullet}
\newcommand{\cross}{\boldsymbol\times}
\newcommand{\grad}{\boldsymbol\nabla}
\newcommand{\divergence}{\grad\dotp}
\newcommand{\curl}{\grad\cross}
%\DeclareMathOperator{\divergence}{divergence}
%\DeclareMathOperator{\curl}[1]{\grad\cross #1}


\colorlet{textColor}{black} 
\colorlet{background}{white}
\colorlet{penColor}{blue!50!black} % Color of a curve in a plot
\colorlet{penColor2}{red!50!black}% Color of a curve in a plot
\colorlet{penColor3}{red!50!blue} % Color of a curve in a plot
\colorlet{penColor4}{green!50!black} % Color of a curve in a plot
\colorlet{penColor5}{orange!80!black} % Color of a curve in a plot
\colorlet{fill1}{penColor!20} % Color of fill in a plot
\colorlet{fill2}{penColor2!20} % Color of fill in a plot
\colorlet{fillp}{fill1} % Color of positive area
\colorlet{filln}{penColor2!20} % Color of negative area
\colorlet{fill3}{penColor3!20} % Fill
\colorlet{fill4}{penColor4!20} % Fill
\colorlet{fill5}{penColor5!20} % Fill
\colorlet{gridColor}{gray!50} % Color of grid in a plot

\newcommand{\surfaceColor}{violet}
\newcommand{\surfaceColorTwo}{redyellow}
\newcommand{\sliceColor}{greenyellow}




\pgfmathdeclarefunction{gauss}{2}{% gives gaussian
  \pgfmathparse{1/(#2*sqrt(2*pi))*exp(-((x-#1)^2)/(2*#2^2))}%
}


%%%%%%%%%%%%%
%% Vectors
%%%%%%%%%%%%%

%% Simple horiz vectors
\renewcommand{\vector}[1]{\left\langle #1\right\rangle}


%% %% Complex Horiz Vectors with angle brackets
%% \makeatletter
%% \renewcommand{\vector}[2][ , ]{\left\langle%
%%   \def\nextitem{\def\nextitem{#1}}%
%%   \@for \el:=#2\do{\nextitem\el}\right\rangle%
%% }
%% \makeatother

%% %% Vertical Vectors
%% \def\vector#1{\begin{bmatrix}\vecListA#1,,\end{bmatrix}}
%% \def\vecListA#1,{\if,#1,\else #1\cr \expandafter \vecListA \fi}

%%%%%%%%%%%%%
%% End of vectors
%%%%%%%%%%%%%

%\newcommand{\fullwidth}{}
%\newcommand{\normalwidth}{}



%% makes a snazzy t-chart for evaluating functions
%\newenvironment{tchart}{\rowcolors{2}{}{background!90!textColor}\array}{\endarray}

%%This is to help with formatting on future title pages.
\newenvironment{sectionOutcomes}{}{} 



%% Flowchart stuff
%\tikzstyle{startstop} = [rectangle, rounded corners, minimum width=3cm, minimum height=1cm,text centered, draw=black]
%\tikzstyle{question} = [rectangle, minimum width=3cm, minimum height=1cm, text centered, draw=black]
%\tikzstyle{decision} = [trapezium, trapezium left angle=70, trapezium right angle=110, minimum width=3cm, minimum height=1cm, text centered, draw=black]
%\tikzstyle{question} = [rectangle, rounded corners, minimum width=3cm, minimum height=1cm,text centered, draw=black]
%\tikzstyle{process} = [rectangle, minimum width=3cm, minimum height=1cm, text centered, draw=black]
%\tikzstyle{decision} = [trapezium, trapezium left angle=70, trapezium right angle=110, minimum width=3cm, minimum height=1cm, text centered, draw=black]


\outcome{Solve basic related rates word problems.}
\outcome{Understand the process of solving related rates problems.}
\outcome{Calculate derivatives of expressions with multiple variables implicitly.}

\title[Dig-In:]{More than one rate}

\begin{document}
\begin{abstract}
  Here we work abstract related rates problems.
\end{abstract}
\maketitle


Suppose we have two variables $x$ and $y$ which are both changing with
respect to time.  A \textit{related rates} problem is a problem where
we know one rate at a given instant, and wish to find the other.

Here the chain rule is key: If $y$ is written in terms of $x$, and we
are given $\dd[x]{t}$, then it is easy to find $\dd[y]{t}$ using the
chain rule:
\[
\dd[y]{t}=y'(x(t))\cdot x'(t).
\]
In many cases, particularly the interesting ones, our functions will
be related in some other way. Nevertheless, in each case we'll use the
power of the chain rule to help us find the desired rate. In this
section, we will work several abstract examples, so we can emphasize
the mathematical concepts involved. In each of the examples below, we
will follow essentially the same plan of attack:

\begin{description}
\item[\textbf{Draw a picture.}] If possible, draw a schematic picture
  with all the relevant information.
\item[\textbf{Find equations.}] We want equations that relate all
  relevant functions.
\item[\textbf{Differentiate the equations.}] Here we will often use
  implicit differentiation.
\item[\textbf{Evaluate and solve.}] Evaluate
  each equation at all known desired values and solve for the relevant
  rate.
\end{description}


\section{Formulas}

One way to combine several functions is with a known formula.

\begin{example}
  Imagine an expanding circle. If we know that the perimeter is
  expanding at a rate of $4$ m/s, what rate is the area changing
  when the radius is $3$ meters?
  \begin{explanation}
    To start, we \textbf{draw a picture}.
    \begin{image}
      \begin{tikzpicture}
        \draw [penColor, very thick] (0,0) circle [radius=2];
        \draw [penColor] (0,0) -- (2,0);
        \node [below,penColor] at (1,0) {$r=3$ m};
        \node [penColor,left] at (-1.5,1.42) {$P'(t) = 4$ m/s};
        \node [penColor, right] at (1.5,-1.42) {$A = \pi\cdot r^2$};
      \end{tikzpicture}
    \end{image}
    We must \textbf{find equations} that combine relevant
    functions. Here we use the common formulas for perimeter and area
    \[
    P = \answer[given]{2\cdot \pi \cdot r}
    \qquad\text{and}\qquad
    A = \answer[given]{\pi \cdot r^2}.
    \]
    Next we imagine that $A$, $r$, and $P$ are functions of time
    \[
    P(t) = 2\cdot \pi \cdot r(t)
    \qquad\text{and}\qquad
    A(t) = \pi \cdot r(t)^2.
    \]
    and we \textbf{differentiate the equations} using implicit
    differentiation, treating all functions as functions of $t$
    \[
    P'(t) = 2\cdot \pi\cdot r'(t)
    \qquad\text{and}\qquad
    A'(t) = 2\cdot \pi\cdot r(t) \cdot r'(t).
    \]
    Now we \textbf{evaluate and solve}. We know $P'(t) =
    \answer[given]{4}$ and that $r(t) = \answer[given]{3}$. Hence our
    equations become
    \[
    4 = 2\cdot \pi\cdot r'(t)
    \qquad\text{and}\qquad
    A'(t) = 2\cdot \pi\cdot 3 \cdot r'(t).
    \]
    We see that
    \begin{align*}
      4 &= 2\cdot \pi\cdot r'(t)\\
      2/\pi &= r'(t).
    \end{align*}
    and now that
    \begin{align*}
      A'(t) &= 2\cdot \pi\cdot 3 \cdot 2/\pi\\
      &=\answer[given]{12}.
    \end{align*}
    Hence the area is expanding at a rate of $12$ m/s.
  \end{explanation}
\end{example}


%%BADBAD
%% There are a number of common formulas that arise in related rates
%% problems.
%%
%% It might be nice to add a little list of common area/volume formulas.



\section{Right triangles}

A common way to combine functions is through facts related to right
triangles.


\begin{example}
  Imagine an expanding right triangle. If one leg has a fixed length
  of $3$ m, one leg is increasing with a rate of $2$ m/s, and the
  hypotenuse is expanding to accommodate the expanding leg, at what
  rate is the hypotenuse expanding when both legs are $3$ m long?
  \begin{explanation}
    To start, we \textbf{draw a picture}.
    \begin{image}
      \begin{tikzpicture}
        \coordinate (A) at (0,2);
        \coordinate (B) at (0,5);
        \coordinate (C) at (6.5,2);
        \tkzMarkRightAngle(C,A,B)
        \tkzDefMidPoint(A,B) \tkzGetPoint{a}
        \tkzDefMidPoint(A,C) \tkzGetPoint{b}
        \tkzDefMidPoint(B,C) \tkzGetPoint{c}
        \draw (A)--(B)--(C)--cycle;
        \tkzLabelPoints[above](c)
        \tkzLabelPoints[above](b)
        %\tkzLabelPoints[left](a)
        \node [left] at (a) {$a = 3$};
        \node [below] at (b) {$b'(t) = 2$};
      \end{tikzpicture}
    \end{image}

    We must \textbf{find equations} that combines relevant
    functions. Here we use the Pythagorean Theorem.
    \[
    c^2 = a^2 + b^2
    \]
    Imagining $c$ and $b$ as being functions of time
    \[
    c(t)^2 = a^2 + b(t)^2
    \]
    we are now able to \textbf{differentiate the equation} using
    implicit differentiation, treating all functions as functions of
    $t$, note $a$ is constant,
    \[
    2\cdot c(t)\cdot c'(t) = 2\cdot b(t)\cdot b'(t).
    \]
    Now we \textbf{evaluate and solve}. We
    know that $b'(t) = 2$ and that $b(t) = 3$
    \[
    2\cdot c(t)\cdot c'(t) = \answer[given]{12}
    \]
    However, we still need to know $c(t)$ when $b(t) = 3$. Here we use
    the Pythagorean Theorem,
    \begin{align*}
    c(t)^2 &= 3^2 + 3^2\\
    &=\answer[given]{18},
    \end{align*}
    and so we see that $c(t) = 3\sqrt{2}$. We may now write
    \begin{align*}
      6\sqrt{2}\cdot c'(t) &= 12 \\
      c'(t) &= \sqrt{2}.
    \end{align*}
    Hence $c(t)$ is growing at a rate of $\answer[given]{\sqrt{2}}$ m/s.
  \end{explanation}
\end{example}


\section{Angular rates}


We can also investigate problems involving angular rates.

\begin{example}
  Imagine an expanding right triangle. If one leg has a fixed length
  of $3$ m, one leg is increasing with a rate of $2$ m/s, and the
  hypotenuse is expanding to accommodate the expanding leg, at what
  rate is the angle opposite the fixed leg changing when both legs
  are $3$ m long?
  \begin{explanation}
    To start, we \textbf{draw a picture}.
    \begin{image}
      \begin{tikzpicture}
        \coordinate (A) at (0,2);
        \coordinate (B) at (0,5);
        \coordinate (C) at (6.5,2);
        \tkzMarkRightAngle(C,A,B)
        \tkzMarkAngle[size=1.2cm,thin](B,C,A)
        \tkzLabelAngle[pos = 1](B,C,A){$\theta$}
        \tkzDefMidPoint(A,B) \tkzGetPoint{a}
        \tkzDefMidPoint(A,C) \tkzGetPoint{b}
        \tkzDefMidPoint(B,C) \tkzGetPoint{c}
        \draw (A)--(B)--(C)--cycle;
        \tkzLabelPoints[above](c)
        \tkzLabelPoints[above](b)
        %\tkzLabelPoints[left](a)
        \node [left] at (a) {$a = 3$};
        \node [below] at (b) {$b'(t) = 2$};
      \end{tikzpicture}
    \end{image} 

    We must \textbf{find equations} that combines relevant
    functions. Here we note that
    \[
    \tan(\theta) = \answer[given]{\frac{a}{b}}
    \]
    Imagining $\theta$ and $b$ as being functions of time
    \[
    \tan(\theta(t)) = \frac{a}{b(t)}
    \]
    we are now able to \textbf{differentiate the equation} using
    implicit differentiation, treating all functions as functions of
    $t$, note $a$ is constant,
    \[
    \sec^2(\theta(t))\theta'(t) = \frac{-a\cdot b'(t)}{b(t)^2}.
    \]
    Now we \textbf{evaluate and solve}.  We
    know that $a=3$, $b'(t) = 2$, and that $b(t) = 3$
    \begin{align*}
    \sec^2(\theta(t))\cdot \theta'(t) &= \frac{-3\cdot 2}{3^2}\\
    &= \frac{-6}{9}\\
    &= \frac{-2}{3}.
    \end{align*}
    However, we still need to know $\sec^2(\theta)$. Here we use the
    Pythagorean Theorem,
    \begin{align*}
    c^2(t) &= 3^2 + 3^2\\
    &=\answer[given]{18},
    \end{align*}
    and so we see that $c(t) = 3\sqrt{2}$. Now
    \begin{align*}
      \sec^2(\theta) &= \frac{\mathrm{hypotenuse}^2}{\mathrm{adjacent}^2}\\
      &= \frac{\left(3\sqrt{2}\right)^2}{3^2}\\
      &= \answer[given]{2}.
    \end{align*}
    Hence
    \begin{align*}
      \sec^2(\theta(t))\cdot \theta'(t) &= \frac{-2}{3}\\
      2\cdot \theta'(t) &= \frac{-2}{3}\\
      \theta'(t) &= \frac{-1}{3}.
    \end{align*}
    So when $a=b=3$, the angle is changing at $\answer[given]{-1/3}$
    radians per second.
  \end{explanation}
\end{example}



\section{Similar triangles}

Finally, facts about similar triangles are often useful when solving
related rates problems.

\begin{example}
  Imagine two right triangles that share an angle:
  \begin{image}
    \begin{tikzpicture}
      \coordinate (A) at (6,2);
      \coordinate (B) at (6,5);
      \coordinate (C) at (0,2);
      \coordinate (D) at (4,2);
      \coordinate (E) at (4,4);
      \tkzMarkRightAngle(C,A,B)
      \tkzMarkRightAngle(C,D,E)
      \tkzDefMidPoint(A,B) \tkzGetPoint{a}
      \tkzDefMidPoint(A,C) \tkzGetPoint{b}
      \tkzDefMidPoint(D,C) \tkzGetPoint{x}
      \draw[decoration={brace,mirror,raise=.2cm},decorate,thin] (0,2)--(6,2);
      \draw[decoration={brace,mirror,raise=.2cm},decorate,thin] (6,2)--(6,5);
      \draw[dashed] (A)--(B)--(C)--cycle;
      \draw[very thick] (D)--(E)--(C)--cycle;
      \tkzLabelPoints[above](x)
      \node at (3,2-.5) {$6$};
      \node at (6+.5,3.5) {$3$};
    \end{tikzpicture}
  \end{image}
  If $x$ is growing from the vertex with a rate of $3$ m/s, what rate
  is the area of the smaller triangle changing when $x = 5$?
  \begin{explanation}
    Despite the fact that a nice picture is given, we should start as
    we always do and \textbf{draw a picture}. Note, we've added
    information to the picture:
    \begin{image}
      \begin{tikzpicture}
        \coordinate (A) at (6,2);
        \coordinate (B) at (6,5);
        \coordinate (C) at (0,2);
        \coordinate (D) at (4,2);
        \coordinate (E) at (4,4);
        \tkzMarkRightAngle(C,A,B)
        \tkzMarkRightAngle(C,D,E)
      \tkzDefMidPoint(A,B) \tkzGetPoint{a}
      \tkzDefMidPoint(A,C) \tkzGetPoint{b}
      \tkzDefMidPoint(D,C) \tkzGetPoint{x}
      \tkzDefMidPoint(D,E) \tkzGetPoint{h}
      \draw[decoration={brace,mirror,raise=.2cm},decorate,thin] (0,2)--(6,2);
      \draw[decoration={brace,mirror,raise=.2cm},decorate,thin] (6,2)--(6,5);
      \draw[dashed] (A)--(B)--(C)--cycle;
      \draw[very thick] (D)--(E)--(C)--cycle;
      \tkzLabelPoints[above](x)
      \tkzLabelPoints[right](h)
      \node at (3,2-.5) {$6$};
      \node at (6+.5,3.5) {$3$};
    \end{tikzpicture}
  \end{image}


    We must \textbf{find equations} that combines relevant
    functions. In this case there are two. The first is the formula
    for the area of a triangle:
    \[
    A = \answer[given]{(1/2) \cdot x \cdot h}
    \]
    The second uses the fact that the larger triangle is similar to
    the smaller triangle, meaning that the proportions of the sides
    are the same,
    \[
    \frac{x}{h} = \answer[given]{\frac{6}{3}}\qquad\text{so}\qquad x =
    \answer[given]{2}\cdot h
    \]
    Imagining $A$, $x$, and $h$ as functions of time we may write
    \[
    A(t) = (1/2) \cdot x(t) \cdot h(t) \qquad\text{and}\qquad x(t) =
    2\cdot h(t).
    \]
    We are now able to \textbf{differentiate the equations} using
    implicit differentiation, treating all functions as functions of
    $t$,
    \begin{align*}
      A'(t) &= (1/2) \cdot x'(t) \cdot h(t) +  (1/2) \cdot x(t) \cdot h'(t),\\
      x'(t) &= 2\cdot h'(t).
    \end{align*}
    Now we \textbf{evaluate and solve}. We
    know that $x(t) = 5$ and that $x'(t) = 3$. Since
    \[
    5=x(t) = 2\cdot h(t)\qquad\text{and}\qquad 3=x'(t)= 2\cdot h'(t)
    \]
    we see that $h(t) = 5/2$ and $h'(t) = 3/2$. Hence
    \begin{align*}
      A'(t) &= (1/2) \cdot 3 \cdot (5/2) + (1/2) \cdot 5 \cdot (3/2)\\
      &= 15/4 + 15/4\\
      &= \answer[given]{15/2}.
    \end{align*}
    Hence the area is changing at a rate of $15/2$
    $\text{m}^2/\text{s}$.
  \end{explanation}
\end{example}
\end{document}
