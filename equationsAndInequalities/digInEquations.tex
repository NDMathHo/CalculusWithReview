\documentclass{ximera}

%\usepackage{todonotes}

\newcommand{\todo}{}

\usepackage{tkz-euclide}
\tikzset{>=stealth} %% cool arrow head
\tikzset{shorten <>/.style={ shorten >=#1, shorten <=#1 } } %% allows shorter vectors

\usepackage{tkz-tab}  %% sign charts
\usetikzlibrary{decorations.pathreplacing} 

\usetikzlibrary{backgrounds} %% for boxes around graphs
\usetikzlibrary{shapes,positioning}  %% Clouds and stars
\usetikzlibrary{matrix} %% for matrix
\usepgfplotslibrary{polar} %% for polar plots
\usetkzobj{all}
\usepackage[makeroom]{cancel} %% for strike outs
%\usepackage{mathtools} %% for pretty underbrace % Breaks Ximera
\usepackage{multicol}

\usepackage{polynom}



\usepackage{array}
\setlength{\extrarowheight}{+.1cm}   
\newdimen\digitwidth
\settowidth\digitwidth{9}
\def\divrule#1#2{
\noalign{\moveright#1\digitwidth
\vbox{\hrule width#2\digitwidth}}}





\newcommand{\RR}{\mathbb R}
\newcommand{\R}{\mathbb R}
\newcommand{\N}{\mathbb N}
\newcommand{\Z}{\mathbb Z}

%\renewcommand{\d}{\,d\!}
\renewcommand{\d}{\mathop{}\!d}
\newcommand{\dd}[2][]{\frac{\d #1}{\d #2}}
\newcommand{\pp}[2][]{\frac{\partial #1}{\partial #2}}
\renewcommand{\l}{\ell}
\newcommand{\ddx}{\frac{d}{\d x}}

\newcommand{\zeroOverZero}{\ensuremath{\boldsymbol{\tfrac{0}{0}}}}
\newcommand{\inftyOverInfty}{\ensuremath{\boldsymbol{\tfrac{\infty}{\infty}}}}
\newcommand{\zeroOverInfty}{\ensuremath{\boldsymbol{\tfrac{0}{\infty}}}}
\newcommand{\zeroTimesInfty}{\ensuremath{\small\boldsymbol{0\cdot \infty}}}
\newcommand{\inftyMinusInfty}{\ensuremath{\small\boldsymbol{\infty - \infty}}}
\newcommand{\oneToInfty}{\ensuremath{\boldsymbol{1^\infty}}}
\newcommand{\zeroToZero}{\ensuremath{\boldsymbol{0^0}}}
\newcommand{\inftyToZero}{\ensuremath{\boldsymbol{\infty^0}}}



\newcommand{\numOverZero}{\ensuremath{\boldsymbol{\tfrac{\#}{0}}}}
\newcommand{\dfn}{\textbf}
%\newcommand{\unit}{\,\mathrm}
\newcommand{\unit}{\mathop{}\!\mathrm}
\newcommand{\eval}[1]{\bigg[ #1 \bigg]}
\newcommand{\seq}[1]{\left( #1 \right)}
\renewcommand{\epsilon}{\varepsilon}
\renewcommand{\iff}{\Leftrightarrow}

\DeclareMathOperator{\arccot}{arccot}
\DeclareMathOperator{\arcsec}{arcsec}
\DeclareMathOperator{\arccsc}{arccsc}
\DeclareMathOperator{\si}{Si}
\DeclareMathOperator{\proj}{proj}
\DeclareMathOperator{\scal}{scal}


\newcommand{\tightoverset}[2]{% for arrow vec
  \mathop{#2}\limits^{\vbox to -.5ex{\kern-0.75ex\hbox{$#1$}\vss}}}
\newcommand{\arrowvec}[1]{\tightoverset{\scriptstyle\rightharpoonup}{#1}}
\renewcommand{\vec}{\mathbf}
\newcommand{\veci}{\vec{i}}
\newcommand{\vecj}{\vec{j}}
\newcommand{\veck}{\vec{k}}
\newcommand{\vecl}{\boldsymbol{\l}}

\newcommand{\dotp}{\bullet}
\newcommand{\cross}{\boldsymbol\times}
\newcommand{\grad}{\boldsymbol\nabla}
\newcommand{\divergence}{\grad\dotp}
\newcommand{\curl}{\grad\cross}
%\DeclareMathOperator{\divergence}{divergence}
%\DeclareMathOperator{\curl}[1]{\grad\cross #1}


\colorlet{textColor}{black} 
\colorlet{background}{white}
\colorlet{penColor}{blue!50!black} % Color of a curve in a plot
\colorlet{penColor2}{red!50!black}% Color of a curve in a plot
\colorlet{penColor3}{red!50!blue} % Color of a curve in a plot
\colorlet{penColor4}{green!50!black} % Color of a curve in a plot
\colorlet{penColor5}{orange!80!black} % Color of a curve in a plot
\colorlet{fill1}{penColor!20} % Color of fill in a plot
\colorlet{fill2}{penColor2!20} % Color of fill in a plot
\colorlet{fillp}{fill1} % Color of positive area
\colorlet{filln}{penColor2!20} % Color of negative area
\colorlet{fill3}{penColor3!20} % Fill
\colorlet{fill4}{penColor4!20} % Fill
\colorlet{fill5}{penColor5!20} % Fill
\colorlet{gridColor}{gray!50} % Color of grid in a plot

\newcommand{\surfaceColor}{violet}
\newcommand{\surfaceColorTwo}{redyellow}
\newcommand{\sliceColor}{greenyellow}




\pgfmathdeclarefunction{gauss}{2}{% gives gaussian
  \pgfmathparse{1/(#2*sqrt(2*pi))*exp(-((x-#1)^2)/(2*#2^2))}%
}


%%%%%%%%%%%%%
%% Vectors
%%%%%%%%%%%%%

%% Simple horiz vectors
\renewcommand{\vector}[1]{\left\langle #1\right\rangle}


%% %% Complex Horiz Vectors with angle brackets
%% \makeatletter
%% \renewcommand{\vector}[2][ , ]{\left\langle%
%%   \def\nextitem{\def\nextitem{#1}}%
%%   \@for \el:=#2\do{\nextitem\el}\right\rangle%
%% }
%% \makeatother

%% %% Vertical Vectors
%% \def\vector#1{\begin{bmatrix}\vecListA#1,,\end{bmatrix}}
%% \def\vecListA#1,{\if,#1,\else #1\cr \expandafter \vecListA \fi}

%%%%%%%%%%%%%
%% End of vectors
%%%%%%%%%%%%%

%\newcommand{\fullwidth}{}
%\newcommand{\normalwidth}{}



%% makes a snazzy t-chart for evaluating functions
%\newenvironment{tchart}{\rowcolors{2}{}{background!90!textColor}\array}{\endarray}

%%This is to help with formatting on future title pages.
\newenvironment{sectionOutcomes}{}{} 



%% Flowchart stuff
%\tikzstyle{startstop} = [rectangle, rounded corners, minimum width=3cm, minimum height=1cm,text centered, draw=black]
%\tikzstyle{question} = [rectangle, minimum width=3cm, minimum height=1cm, text centered, draw=black]
%\tikzstyle{decision} = [trapezium, trapezium left angle=70, trapezium right angle=110, minimum width=3cm, minimum height=1cm, text centered, draw=black]
%\tikzstyle{question} = [rectangle, rounded corners, minimum width=3cm, minimum height=1cm,text centered, draw=black]
%\tikzstyle{process} = [rectangle, minimum width=3cm, minimum height=1cm, text centered, draw=black]
%\tikzstyle{decision} = [trapezium, trapezium left angle=70, trapezium right angle=110, minimum width=3cm, minimum height=1cm, text centered, draw=black]


\outcome{Solve linear equations.}
\outcome{Solve quadratic equations.}
\outcome{Solve equations by factoring.}


\title[Dig-In:]{Equations}
\begin{document}
\begin{abstract}
  We discuss solving equations.
\end{abstract}
\maketitle

An equation is a statement expressing the equality between two quantities.  This means that an equation will always have a single equals sign.


\begin{example}
  From Devyn's question above, $\displaystyle \frac{74 + 84 + x}{3}$ is the average of his three exam 
  scores, with the variable $x$ representing the third unknown exam score.  What does the equation
  $\displaystyle \dfrac{74 + 84 + x}{3} = 82$ mean in this setting?
  
  \begin{explanation}
    This is an equation, stating that the average of Devyn's three exam scores is 82.
  \end{explanation}
\end{example}

Usually, when dealing with an equation, we will be looking for values that make the equation true.
A solution to an equation is a value that, when substituted in for the variables, yield a true number statement.

\begin{example}
	$\displaystyle \dfrac{74 + 84 + 88}{3} = 82$, so $x = 88$ is a solution of the equation
	$\displaystyle \dfrac{74 + 84 + x}{3} = 82$.
\end{example}

When we are asked to \emph{solve an equation}, we are being asked to find all solutions.
Exactly how to do that depends on the particular equation involved.

A \emph{linear equation in $x$} is an equation which is equivalent to one with the form
$a x + b = 0$, where $a$ and $b$ are constants, with $a \neq 0$.


To solve a linear equation, we isolate the $x$-term and divide by the coefficient.
\begin{example}
	Solve the linear equation $\displaystyle \dfrac{x}{3} - 9 = 4\left( 2x + \dfrac{5}{2} \right)$.

	\begin{explanation}
		This doesn't really look like $ax + b = 0$ when written like this.  Let's start by distributing the $4$ into 
		the parentheses, then combining like terms.
		\begin{align*}
			\frac{x}{3} - 9 &= 4\left( 2x + \dfrac{5}{2} \right) \\
			\frac{x}{3} - 9 &= 8x + 10
		\end{align*}
		That looks better.  From here, we'll move all the $x$-terms to the right side of the equation, and all of
		the constant terms to the left.
		\begin{align*}
			- 9 - 10 &= 8x - \frac{x}{3}\\
			-19 &= \frac{23}{3} x \\
			x &= -19 \cdot \frac{3}{23} = -\frac{57}{23}
		\end{align*}
	\end{explanation}	
\end{example}


\begin{question}
 	Solve the linear equation $\displaystyle \dfrac{4}{5}\left(x+2\right) - 3 = \dfrac{x-1}{2}$. 
 	\begin{hint}
    		It may be easier to clear fractions first.
	\end{hint}
  	\begin{prompt}
    		x = $\answer{3}$.
  	\end{prompt}
\end{question}


A \emph{quadratic equation in $x$} is an equation which is equivalent to one with the form
	$ax^2 + bx + c = 0$, where $a$, $b$, and $c$ are constants, with $a \neq 0$.

There are three major techniques you are probably familiar with for solving quadratic equations:
\begin{itemize}
	\item Factoring.
	\item Completing the Square.
	\item Quadratic Formula.
\end{itemize}

Each of these methods are important.  Factoring is vital, because it is a valid approach to solve nearly any type of equation.  Completing
the Square is a technique that becomes useful when we need to rewrite certain types of expressions.  The Quadratic Formula will always
work, but has some limitations.

\begin{example}
	Solve the quadratic equation
	\[ 2x^2 +5x = 27 x - 60. \]
	\begin{explanation}
		We'll start by writing this in it's standard form: $2x^2 - 22x + 60 = 0$.  Notice how there is a common factor of 2? Let's divide by 2.
		
		$x^2 - 11x + 30 = 0$.  Any of the above methods will work here, so let's try factoring.  What numbers add to $-11$ and multiply to $30$?
		$-5$ and $-6$ do. That gives us:
		\begin{align*}
			x^2 - 11x + 30 &= 0\\
			(x-5)(x-6) &= 0
		\end{align*}
		Either $x-5 = 0$ (giving us $x=5$) or $x-6 = 0$ (giving us $x=6$).  The two solutions are $x = 5, 6$.
	\end{explanation}
\end{example}	

\begin{example}
	Solve the quadratic equation
	\[  x^2 = 6x -4.\]
	\begin{explanation}
		Again, we'll write it in standard form: $x^2 - 6x + 4 = 0$.  The quadratic $x^2 -6x+4$ does not factor, so we'll complete the square instead.
		
		Start by moving the constant term to the other side
		\[ x^2 - 6x = -4. \]
		This has $b = -6$.  That means $\displaystyle \dfrac{b}{2} = -3$ and $\displaystyle \left( \dfrac{b}{2}\right)^2 = 9$.  Add $9$ to both sides.
		\[ x^2 - 6x + 9 = -4 + 9 = 5. \]
		The left-hand side is now a perfect square, $\left(x-3\right)^2 = x^2 - 6x + 9$.  We can then solve using square roots.
		\begin{align*}
			x^2 - 6x + 9 &=  5\\
			\left( x-3 \right)^2 &= 5\\
			x-3 = \pm \sqrt{5} \\
			x = 3 \pm \sqrt{5}.
		\end{align*}
	\end{explanation}
\end{example}
If you had used the quadratic formula, $\displaystyle x = \dfrac{-b \pm \sqrt{b^2-4ac}}{2a}$, instead of factoring or completing the square above, you would have found the same solutions.

\begin{question}
	Solve the quadratic equation $\displaystyle x^2 + 4 = 4\left(x+2\right)$.
	\begin{multipleChoice}
		\choice{$x = \pm 2$}
		\choice[correct]{$x=2\pm 2\sqrt{2}$}
		\choice{$x = -2, -4$}
		\choice{none of the above}
  \end{multipleChoice}
\end{question}

Factoring is a process that helps us solve more than just quadratic equations, as long as we first get one side of the equation equal to zero.
\begin{example}
	Solve the equation
	\[ 2x^2 \left( x-4 \right) \left( 2x-5\right)^3 = 0. \]
	\begin{explanation}
		The only way for a product of real numbers to be zero, is for one of the factors to have been zero.
		Here, that means either $2x^2 = 0$, or $x-4 = 0$ or $(2x-5)^3 = 0$.
		This gives us three equations, each substantially less complicated than the original one, to solve.  By solving them, we find
		$x = 0, 4, \frac{5}{2}$.
	\end{explanation}
\end{example}

\end{document}
