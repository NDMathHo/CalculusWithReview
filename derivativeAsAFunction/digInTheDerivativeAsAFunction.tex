\documentclass{ximera}

%\usepackage{todonotes}

\newcommand{\todo}{}

\usepackage{tkz-euclide}
\tikzset{>=stealth} %% cool arrow head
\tikzset{shorten <>/.style={ shorten >=#1, shorten <=#1 } } %% allows shorter vectors

\usepackage{tkz-tab}  %% sign charts
\usetikzlibrary{decorations.pathreplacing} 

\usetikzlibrary{backgrounds} %% for boxes around graphs
\usetikzlibrary{shapes,positioning}  %% Clouds and stars
\usetikzlibrary{matrix} %% for matrix
\usepgfplotslibrary{polar} %% for polar plots
\usetkzobj{all}
\usepackage[makeroom]{cancel} %% for strike outs
%\usepackage{mathtools} %% for pretty underbrace % Breaks Ximera
\usepackage{multicol}

\usepackage{polynom}



\usepackage{array}
\setlength{\extrarowheight}{+.1cm}   
\newdimen\digitwidth
\settowidth\digitwidth{9}
\def\divrule#1#2{
\noalign{\moveright#1\digitwidth
\vbox{\hrule width#2\digitwidth}}}





\newcommand{\RR}{\mathbb R}
\newcommand{\R}{\mathbb R}
\newcommand{\N}{\mathbb N}
\newcommand{\Z}{\mathbb Z}

%\renewcommand{\d}{\,d\!}
\renewcommand{\d}{\mathop{}\!d}
\newcommand{\dd}[2][]{\frac{\d #1}{\d #2}}
\newcommand{\pp}[2][]{\frac{\partial #1}{\partial #2}}
\renewcommand{\l}{\ell}
\newcommand{\ddx}{\frac{d}{\d x}}

\newcommand{\zeroOverZero}{\ensuremath{\boldsymbol{\tfrac{0}{0}}}}
\newcommand{\inftyOverInfty}{\ensuremath{\boldsymbol{\tfrac{\infty}{\infty}}}}
\newcommand{\zeroOverInfty}{\ensuremath{\boldsymbol{\tfrac{0}{\infty}}}}
\newcommand{\zeroTimesInfty}{\ensuremath{\small\boldsymbol{0\cdot \infty}}}
\newcommand{\inftyMinusInfty}{\ensuremath{\small\boldsymbol{\infty - \infty}}}
\newcommand{\oneToInfty}{\ensuremath{\boldsymbol{1^\infty}}}
\newcommand{\zeroToZero}{\ensuremath{\boldsymbol{0^0}}}
\newcommand{\inftyToZero}{\ensuremath{\boldsymbol{\infty^0}}}



\newcommand{\numOverZero}{\ensuremath{\boldsymbol{\tfrac{\#}{0}}}}
\newcommand{\dfn}{\textbf}
%\newcommand{\unit}{\,\mathrm}
\newcommand{\unit}{\mathop{}\!\mathrm}
\newcommand{\eval}[1]{\bigg[ #1 \bigg]}
\newcommand{\seq}[1]{\left( #1 \right)}
\renewcommand{\epsilon}{\varepsilon}
\renewcommand{\iff}{\Leftrightarrow}

\DeclareMathOperator{\arccot}{arccot}
\DeclareMathOperator{\arcsec}{arcsec}
\DeclareMathOperator{\arccsc}{arccsc}
\DeclareMathOperator{\si}{Si}
\DeclareMathOperator{\proj}{proj}
\DeclareMathOperator{\scal}{scal}


\newcommand{\tightoverset}[2]{% for arrow vec
  \mathop{#2}\limits^{\vbox to -.5ex{\kern-0.75ex\hbox{$#1$}\vss}}}
\newcommand{\arrowvec}[1]{\tightoverset{\scriptstyle\rightharpoonup}{#1}}
\renewcommand{\vec}{\mathbf}
\newcommand{\veci}{\vec{i}}
\newcommand{\vecj}{\vec{j}}
\newcommand{\veck}{\vec{k}}
\newcommand{\vecl}{\boldsymbol{\l}}

\newcommand{\dotp}{\bullet}
\newcommand{\cross}{\boldsymbol\times}
\newcommand{\grad}{\boldsymbol\nabla}
\newcommand{\divergence}{\grad\dotp}
\newcommand{\curl}{\grad\cross}
%\DeclareMathOperator{\divergence}{divergence}
%\DeclareMathOperator{\curl}[1]{\grad\cross #1}


\colorlet{textColor}{black} 
\colorlet{background}{white}
\colorlet{penColor}{blue!50!black} % Color of a curve in a plot
\colorlet{penColor2}{red!50!black}% Color of a curve in a plot
\colorlet{penColor3}{red!50!blue} % Color of a curve in a plot
\colorlet{penColor4}{green!50!black} % Color of a curve in a plot
\colorlet{penColor5}{orange!80!black} % Color of a curve in a plot
\colorlet{fill1}{penColor!20} % Color of fill in a plot
\colorlet{fill2}{penColor2!20} % Color of fill in a plot
\colorlet{fillp}{fill1} % Color of positive area
\colorlet{filln}{penColor2!20} % Color of negative area
\colorlet{fill3}{penColor3!20} % Fill
\colorlet{fill4}{penColor4!20} % Fill
\colorlet{fill5}{penColor5!20} % Fill
\colorlet{gridColor}{gray!50} % Color of grid in a plot

\newcommand{\surfaceColor}{violet}
\newcommand{\surfaceColorTwo}{redyellow}
\newcommand{\sliceColor}{greenyellow}




\pgfmathdeclarefunction{gauss}{2}{% gives gaussian
  \pgfmathparse{1/(#2*sqrt(2*pi))*exp(-((x-#1)^2)/(2*#2^2))}%
}


%%%%%%%%%%%%%
%% Vectors
%%%%%%%%%%%%%

%% Simple horiz vectors
\renewcommand{\vector}[1]{\left\langle #1\right\rangle}


%% %% Complex Horiz Vectors with angle brackets
%% \makeatletter
%% \renewcommand{\vector}[2][ , ]{\left\langle%
%%   \def\nextitem{\def\nextitem{#1}}%
%%   \@for \el:=#2\do{\nextitem\el}\right\rangle%
%% }
%% \makeatother

%% %% Vertical Vectors
%% \def\vector#1{\begin{bmatrix}\vecListA#1,,\end{bmatrix}}
%% \def\vecListA#1,{\if,#1,\else #1\cr \expandafter \vecListA \fi}

%%%%%%%%%%%%%
%% End of vectors
%%%%%%%%%%%%%

%\newcommand{\fullwidth}{}
%\newcommand{\normalwidth}{}



%% makes a snazzy t-chart for evaluating functions
%\newenvironment{tchart}{\rowcolors{2}{}{background!90!textColor}\array}{\endarray}

%%This is to help with formatting on future title pages.
\newenvironment{sectionOutcomes}{}{} 



%% Flowchart stuff
%\tikzstyle{startstop} = [rectangle, rounded corners, minimum width=3cm, minimum height=1cm,text centered, draw=black]
%\tikzstyle{question} = [rectangle, minimum width=3cm, minimum height=1cm, text centered, draw=black]
%\tikzstyle{decision} = [trapezium, trapezium left angle=70, trapezium right angle=110, minimum width=3cm, minimum height=1cm, text centered, draw=black]
%\tikzstyle{question} = [rectangle, rounded corners, minimum width=3cm, minimum height=1cm,text centered, draw=black]
%\tikzstyle{process} = [rectangle, minimum width=3cm, minimum height=1cm, text centered, draw=black]
%\tikzstyle{decision} = [trapezium, trapezium left angle=70, trapezium right angle=110, minimum width=3cm, minimum height=1cm, text centered, draw=black]


\outcome{Understand the derivative as a function related to the original
  definition of a function.}
\outcome{Find the derivative function using the limit definition.}
\outcome{Relate the derivative function to the derivative at a point.}
\outcome{Relate the graph of the function to the graph of its derivative.}




\title[Dig-in:]{The derivative as a function}

\begin{document}
\begin{abstract}
Here we study the derivative of a function, as a function, in its own
right.
\end{abstract}
\maketitle

\section{The derivative of a function, as a function}


We know that to find the derivative of a function at a point $x=a$ we
write
\[
f'(a) = \lim_{h\to 0}\frac{f(a+h)-f(a)}{h}.
\]
However, if we replace the given number $a$ with a variable $x$, we now
have
\[
f'(x) = \lim_{h\to 0}\frac{f(x+h)-f(x)}{h}.
\]
This tells us the instantaneous rate of change at any given point $x$.
\begin{warning}
  The notation:
  \begin{quote}
  $f'(a)$ means take the derivative of $f$ first, then evaluate at
    $x=a$.
  \end{quote}
  In other words, given $f$ a function of $x$
  \[
  f'(a) = \eval{\ddx f(x)}_{x=a}.
  \]
\end{warning}
Given a function $f$ from the real numbers to the real numbers, the
derivative $f'$ is also a function from the real numbers to the real
numbers. Understanding the relationship between the \textit{functions}
$f$ and $f'$ helps us understand any situation (real or imagined)
involving changing values. 

\begin{question}
  Let $f(x) = 3x+2$. What is $f'(-1)$?
  \begin{multipleChoice}
    \choice{$f'(-1) = 0$ because $f'(3)$ is a number, and a number corresponds to a horizontal line, which has a slope of zero.}
    \choice[correct]{$f'(-1) = 3$ because $y=f(x)$ is a line with slope $3$.}
    \choice{We cannot solve this problem yet.}
  \end{multipleChoice}
\end{question}


\begin{question}
  Here we see the graph of $f'$. 
  \begin{image}
    \begin{tikzpicture}
      \begin{axis}[
          xmin=-2,xmax=2,ymin=-8,ymax=8,
          axis lines=center,
          ticks=none,
          width=6in,
          height=3in,
          every axis y label/.style={at=(current axis.above origin),anchor=south},
          every axis x label/.style={at=(current axis.right of origin),anchor=west},
        ]        
        %\addplot [very thick,dashed, penColor,smooth, domain=(-2:2)] {x^3+.3*x^2-2*x)};
        \addplot [very thick,penColor,smooth, domain=(-2:2)] {3*x^2+2*.3*x-2)};
      \end{axis}
    \end{tikzpicture}
  \end{image}
  Describe $y=f(x)$ when $f'$ is positive. Describe $y=f(x)$ when $f'$
  is negative.
  \begin{prompt}
    When $f'$ is positive, $y=f(x)$ is \wordChoice{\choice{positive}\choice[correct]{increasing}\choice{negative}\choice{decreasing}}.
    When $f'$ is negative, $y=f(x)$ is \wordChoice{\choice{positive}\choice{increasing}\choice{negative}\choice[correct]{decreasing}}
  \end{prompt}
  \begin{question}
    Which of the following graphs could be $y = f(x)$?
     \begin{multipleChoice}
       \choice{\begin{tikzpicture}[framed,scale=1,baseline=3ex]
           \begin{axis}[
               xmin=-2,xmax=2,ymin=-8,ymax=8,
               axis lines=center,
               ticks=none,
               width=2in,
               height=1in,
               every axis y label/.style={at=(current axis.above origin),anchor=south},
               every axis x label/.style={at=(current axis.right of origin),anchor=west},
             ]        
             \addplot [very thick,penColor,smooth, domain=(-2:2)] {3*x^2+2*.3*x-2)};
           \end{axis}
       \end{tikzpicture}}
       \choice[correct]{\begin{tikzpicture}[framed,scale=1,baseline=3ex]
           \begin{axis}[
               xmin=-2,xmax=2,ymin=-8,ymax=8,
               axis lines=center,
               ticks=none,
               width=2in,
               height=1in,
               every axis y label/.style={at=(current axis.above origin),anchor=south},
               every axis x label/.style={at=(current axis.right of origin),anchor=west},
             ]        
             \addplot [very thick,penColor,smooth, domain=(-2:2)] {x^3+.3*x^2-2*x)};
           \end{axis}
       \end{tikzpicture}}
       \choice{\begin{tikzpicture}[framed,scale=1,baseline=3ex]
           \begin{axis}[
               xmin=-2,xmax=2,ymin=-8,ymax=8,
               axis lines=center,
               ticks=none,
               width=2in,
               height=1in,
               every axis y label/.style={at=(current axis.above origin),anchor=south},
               every axis x label/.style={at=(current axis.right of origin),anchor=west},
             ]        
             \addplot [very thick,penColor,smooth, domain=(-2:2)] {6*x+2*.3)};
           \end{axis}
       \end{tikzpicture}}
     \end{multipleChoice}
  \end{question}
\end{question}



\section{The derivative as a function of functions}

While writing $f'$ is viewing the derivative of $f$ as a function in
its own right, the derivative itself
\[
\ddx
\]
is in fact a function that maps functions to functions,
\begin{align*}
  \ddx x^2 &= 2x\\
  \ddx f(x) &= f'(x).
\end{align*}

\begin{question}
  As a function, is
  \[
  \ddx
  \]
  one-to-one?
  \begin{multipleChoice}
    \choice{yes}
    \choice[correct]{no}
  \end{multipleChoice}
  \begin{feedback}
    Many different functions share the same derivative since the
    derivative records only the slope of the tangent line and not
    the value, or height of the function.
  \end{feedback}
\end{question}

\end{document}

